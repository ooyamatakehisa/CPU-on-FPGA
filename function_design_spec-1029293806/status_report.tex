\documentclass[a4paper,11pt,oneside,openany]{jsarticle}
\usepackage[dvipdfmx]{graphicx}
\usepackage{amsmath,amssymb}
\usepackage{bm}
\usepackage{graphicx}
\usepackage{ascmac}
%
% \setlength{\textwidth}{\fullwidth}
% \setlength{\textheight}{40\baselineskip}
% \addtolength{\textheight}{\topskip}
%\setlength{\voffset}{-0.55in}
\thispagestyle{empty}
%
\begin{document}
\begin{center}

  \vspace*{40mm}
  \huge 計算機科学実験及演習3 \par
  実施状況報告書\\
  \vspace{90mm}
  \Large 提出期限:2019/5/9\\
   提出日: \today \\
  \vspace{15mm}
  \Large グループ番号: 9    \\
   1029293806\hspace{5mm}大山 偉永\par
   1029286786\hspace{5mm}富村 勇貴\par

  \vspace{10mm}
\end{center}
\clearpage
\addtocounter{page}{-1}

\newpage

\section{分担状況}
ここまでの分担は主にALU、MUX、パイプラインレジスタ、加算器、レジスタ等の小さなコンポーネント作成は自分が制御部と自分が作ったパーツを適切な位置に配置しwire等でつなぐ記述をするのがペアの分担であった。その後パイプライン化、フォワーディングユニット$\cdot$ハザードユニットの設計等は大まかには二人でやり実装はペアに行ってもらった。デバッグも二人で気づいたことを言い合いながら行った。こうすることでミスにもすばやく気づくことができ意外にそれぞれで分担をふるより二人で一つに集中して行うほうが早い気もした。この後に時間があれば分岐予測も実装したいがそれも二人で設計は考えて片方が実装するという形になりそうである。またアセンブラからバイナリへの変換プログラムも自分が作った。これは特に応用プログラムを書くときには必要になるがシュミレーションの際にも非常に便利で早めに作っておいてよかった。

\section{最終目標に対する現在の進捗状況}
もともとスケジュールを立てた中間デモまでにパイプライン化を完成させるという目標は現在実装をしている途中であるために果たされないかもしれないが特別遅れているというわけではない。具体的にはフォワーディングユニット$\cdot$ハザードユニットの記述までは終え、コンパイルは通ったところである。レポート提出日にシュミレーションを終わらせフォワーディングユニットを置くフェーズを変える等の小さな変更を加え次の作業に移っていく予定である。ここから速さ性能を求めて高速化を追求するか、他の命令などの拡張をするかはパイプライン化の完成が遅れる可能性もあるので柔軟に今後のスケジュールを考えて決めていく予定である。一応最初に立てた予定ではパイプライン化を中間デモまでになんとか完成させその後動的分岐予測とかいくつかの命令拡張をし実験を終える予定であった。ただし過去のソートコンテストに出た先輩方のアーキテクチャの特徴を見たところ、動的分岐予測を実装している先輩はあまりおらず実装が難しいわりに高速化はそこまでしないコスパの悪い実装になるのか?という気もしているので静的分岐予測に変更する可能性もある。またOoOやスーパースカラ、マルチコア化なども過去に実装している人がいるので検討してみたいが果たしてどれが「コスパ」のいい改良になるのかわからないので少し勉強が必要である。ソートコンテストについては過去のコンテスト結果で10以内くらいに入れそうな速度が実現できたら出てみようかと考えている。ただ未だにタイミング制約等はしていないので少し難しいかなと考えている。

\subsection{プロジェクトSIMPLEのタグ名}
medium


\end{document}
%todo: 全体画像とコントローラ画像の再読込、レジスタから出る信号線の確認l64,6971付近、ブランチ制御モジュールの名前は?branchのやつ逆では??
