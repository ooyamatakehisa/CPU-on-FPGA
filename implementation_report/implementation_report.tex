\documentclass[a4paper,11pt,oneside,openany]{jsarticle}
\usepackage[dvipdfmx]{graphicx}
\usepackage{amsmath,amssymb}
\usepackage{bm}
\usepackage{graphicx}
\usepackage{ascmac}

\thispagestyle{empty}
%
\begin{document}
\begin{center}

  \vspace*{35mm}
  \huge 計算機科学実験及演習3 \par
  実施報告書 \\
  \vspace{90mm}
  \Large 提出期限:2019/6/14\\
   提出日: \today \\
  \vspace{15mm}
  \Large グループ番号: 9    \\
   1029293806\hspace{5mm}大山 偉永\par
   1029286786\hspace{5mm}富村 勇貴\par

  \vspace{10mm}
\end{center}
\clearpage
\addtocounter{page}{-1}

\newpage

\section{目標の達成度}
中間報告時点でのアーキテクチャ検討報告書には目標は「即値オペランドの強化、ブ ランチ命令の一命令化、動的分岐予測も伴うパイプライン化の実装を目標とする。 次に数値目標については LE 数は 3000 以内、周波数は 100 を目指すが速度を向上したいので LE 数に ついてはそこまで言及しない予定である。」と記してあった。まず即値命令はADDIやCMPIを増やしたことで即値オペランドの強化は達成された。ブランチ命令の一命令化、動的分岐予測の実装についてはソートコンテストのアルゴリズムデバッグ作業が予想以上に時間がかかり達成することができず分岐予測も不成立の静的分岐予測となった。LE数は1543なので目標は達成されたが最高周波数についてはFMAX:74MHzと遠く及ば無かった。機能設計仕様書でも書いたとおり様々な策を講じた上でのこの最高周波数でありTAさんにも助言を求めた上でのこの値なのでこれ以上の改善は見込まれなかった。ソートコンテスト上位の班の170MHzなどはどのように実装したのか知りたい。

\section{分担状況}
まず中間デモまでは自分はマルチプレクサやAL 等の細かい部品を作成しアセンブリ言語をバイナリに変換するプログラムを書き、ペアがフォワーディングユニッ ト、ハザー ド検出ユニット、制御部、全てのコンポーネントを繋ぐ作業を行った。中間デモ以降は自分は即値加算命令とジャンプ命令の拡張を行いその後はソートコンテスト用の基数+挿入ソートのアセンブリを書きペアが即値比較命令、in命令、out命令、exec命令を追加した。デバッグについては自分の作業を止めて二人で協力して行った。

\section{最終的な設計データのGitLab上のプロジェクトSIMPLEのタグ名}
  final






\end{document}
